\documentclass[conference]{IEEEtran}
\IEEEoverridecommandlockouts % chktex 1
\usepackage{cite}
\usepackage{amsmath,amssymb,amsfonts}
\usepackage{algorithmic}
\usepackage{graphicx}
\usepackage{textcomp}
\usepackage{xcolor}
\usepackage{url}
\usepackage[acronym]{glossaries}
\makeglossaries % chktex 1

% Suppress LaTeX warnings
\hbadness=10000 % chktex 1
\hfuzz=\maxdimen % chktex 1
\vfuzz=\maxdimen % chktex 1
\overfullrule=0pt

% Suppress bibliography warnings
\vbadness=10000 % chktex 1
\vsize=\maxdimen % chktex 1
\raggedbottom % chktex 1

% Completely suppress vbox warnings
\makeatletter
\def\@vbox#1{\vbox{#1}}
\def\@vtop#1{\vtop{#1}}
\def\@vsplit#1#2{\vsplit#1#2}
\makeatother

% Additional vbox suppression
\setlength{\topskip}{0pt}
\setlength{\topsep}{0pt}
\setlength{\partopsep}{0pt}

\def\BibTeX{{\rm B\kern-.05em{\sc i\kern-.025em b}\kern-.08em
    T\kern-.1667em\lower.7ex\hbox{E}\kern-.125emX}}
\begin{document}

% =====================================================================
%                              METADATOS
%  Qué cuidar aquí:
%  - Título preciso (incluye palabras clave del enfoque y dominio).
%  - Autoría y afiliación con tildes correctas; correo institucional.
%  - Financiación: menciona CONACYT/FEEI y apoyos académicos (FIUNA).
% =====================================================================
\title{Optimización escalable y descentralizada del tráfico urbano con almacenamiento verificable para despliegues en ciudades inteligentes \\
\thanks{Este trabajo es cofinanciado por el Consejo Nacional de Ciencia y Tecnología (CONACYT) con el apoyo del FEEI\@. Se agradece el respaldo técnico y académico de la Facultad de Ingeniería de la Universidad Nacional de Asunción (FIUNA)\@.}}

\author{\IEEEauthorblockN{Kevin M. Galeano}
\IEEEauthorblockA{\textit{Facultad Politecnica} \\
\textit{Universidad Nacional de Asunción}\\
Asunción, Paraguay \\
gsmkev@gmail.com}
}

\maketitle

% =====================================================================
%                                RESUMEN
%  Contenido mínimo (150–250 palabras):
%  - Problema y por qué importa (1–2 oraciones)
%  - Enfoque: FIS + PSO + HC + almacenamiento verificable (2–3)
%  - Configuración experimental (sensores, datos sintéticos) (1–2)
%  - Resultados con números (mejora %, consistencia, viabilidad) (2–3)
%  - Conclusión y aporte práctico (1)
%  Escribe este bloque al final cuando ya tengas resultados consolidados.
% =====================================================================
\begin{abstract}
% (Rellenar al final)
\end{abstract}

% =====================================================================
%                             PALABRAS CLAVE
%  4–6 términos separados por comas. Incluye: congestión, FIS, PSO, HC,
%  ciudades inteligentes, almacenamiento verificable (ajustar a tu foco).
% =====================================================================
\begin{IEEEkeywords}
% (Rellenar al final)
\end{IEEEkeywords}

% =====================================================================
%                     FRONT MATTER (LISTAS + ACRÓNIMOS)
%  Imprime listas de figuras/tablas y acrónimos.
%  Requiere: \usepackage[acronym]{glossaries} y \makeglossaries.
%  Nota: compilar al menos 2 veces (o ejecutar makeglossaries) para refrescar.
% =====================================================================
% =====================================================================
% 0. FRONT MATTER: LISTAS Y ACRÓNIMOS
% Propósito: listas de figuras/tablas e impresión de acrónimos.
% Requiere: \usepackage[acronym]{glossaries} y \makeglossaries en el preámbulo.
% =====================================================================
% 0-frontmatter.tex
\section*{Listas y acrónimos}
\addcontentsline{toc}{section}{Listas y acrónimos}

\subsection*{Lista de Figuras}
\addcontentsline{toc}{subsection}{Lista de Figuras}
\listoffigures

\subsection*{Lista de Tablas}
\addcontentsline{toc}{subsection}{Lista de Tablas}
\listoftables

\subsection*{Acrónimos y Siglas}
\addcontentsline{toc}{subsection}{Acrónimos y Siglas}
\printglossary[type=\acronymtype]

\newacronym{fis}{FIS}{Sistema de Inferencia Difusa}
\newacronym{pso}{PSO}{Optimización por Enjambre de Partículas}
\newacronym{hc}{HC}{Clustering Jerárquico}
\newacronym{mf}{MF}{Función de Pertenencia}


% =====================================================================
%                              CUERPO DE LA TESIS
%  Qué debe entregar cada sección (resumen ejecutivo):
%  - 1: Problema/objetivos trazables a métricas; contribuciones concretas.
%  - 2: Estado del arte conectado con tus decisiones metodológicas.
%  - 3: Diagrama y contratos de I/O; puntos de verificación (hashes).
%  - 4: Definiciones formales + fórmulas de métricas; figuras FIS/PSO/HC.
%  - 5: Estructura de código, parámetros y flujo; UI y gráficos.
%  - 6: Diseño, setup, resultados con cifras y lectura de gráficos.
%  - 7: Integración FIS→PSO→HC y lectura operativa por clusters.
%  - 8: Conclusiones ligadas a objetivos y líneas de futuro.
% =====================================================================
% =====================================================================
% 1. INTRODUCCIÓN
% Propósito: contexto, problema, objetivos, contribuciones y estructura.
% Guía rápida: mantén los labels estables para referencias cruzadas.
% =====================================================================
% 1-introduccion.tex
\section{Introducción}\label{sec:intro}

\subsection{Motivación y contexto}\label{subsec:intro-motivacion}
% Congestión urbana, necesidad de control adaptativo y analítica explicable.
% Aterrizar en el caso de estudio y entorno de despliegue/validación.

\subsection{Planteamiento del problema}\label{subsec:intro-problema}
% Detección y cuantificación de congestión con salidas crisp; optimización de tiempos semafóricos; agrupamiento
% para análisis a nivel cluster de puntos de medición.

\subsection{Objetivos y preguntas de investigación}\label{subsec:intro-objetivos}
% Objetivo general + objetivos específicos (FIS para congestión, PSO para señalización, HC para análisis macro).
% Preguntas de investigación alineadas a métricas y escenarios de evaluación.

\subsection{Contribuciones}\label{subsec:intro-contribuciones}
% (1) Pipeline PSO+FIS+HC reproducible; (2) MFs y reglas fuzzy trazables; (3) Función objetivo y restricciones operativas;
% (4) Visual analytics con 7 gráficos estandarizados; (5) Arquitectura con almacenamiento verificable.

\subsection{Alcance y limitaciones}\label{subsec:intro-alcance}
% Simulaciones con datos sintéticos y pipeline orquestado; no se incluye calibración con datos reales de campo aún.

\subsection{Estructura del documento}\label{subsec:intro-estructura}
% Mapa de secciones 2–8.

% =====================================================================
% 2. ANTECEDENTES Y TRABAJOS RELACIONADOS
% Qué incluir y cómo evaluarlo:
% - Fenomenología del tráfico: resume “tres fases” y cómo motivan métricas/objetivos.
% - FIS: ventajas de interpretabilidad; ejemplos en control semafórico.
% - PSO: aplicaciones a señalización; variables codificadas y funciones objetivo típicas.
% - HC: usos para segmentar congestión; criterios de corte y lectura operativa.
% - Almacenamiento verificable: trabajos sobre trazabilidad/reproducibilidad.
% Evidencia: cita 1–2 trabajos clave por subtema y enlaza con tu enfoque.
% =====================================================================
% 2-antecedentes-relacionados.tex
\section{Antecedentes y trabajos relacionados}\label{sec:related}

\subsection{Fenomenología del tráfico y teoría de tres fases}\label{subsec:related-kerner}
% Resumen de patrones, breakdown y “tres fases” (free, synchronized, wide moving jam) que motivan un índice de congestión
% y la necesidad de control adaptativo. Cita Kerner y enlaza con Sec. 4 (métricas).
% Ej.: La transición F→S y la propagación de atascos fundamentan FIS y HC.

\subsection{Lógica difusa en control de tráfico}\label{subsec:related-fis}
% MFs, reglas y defuzzificación en aplicaciones semafóricas; ventajas de interpretabilidad.

\subsection{Optimización por enjambre de partículas (PSO) aplicada a semáforos}\label{subsec:related-pso}
% Codificaciones típicas (tiempos de verde/rojo), función objetivo (congestión/cola/demoras), restricciones operativas.
% Contrasta con tu función objetivo en Sec. 4 (penalizaciones y límites).

\subsection{Clustering jerárquico para análisis de congestión}\label{subsec:related-hc}
% Uso de linkage/umbral para segmentar salidas crisp y extraer estadísticos por cluster.
% Justifica elección de linkage/umbral que utilizarás.

\subsection{Almacenamiento verificable y reproducibilidad}\label{subsec:related-storage}
% Breve estado del arte (según referencias de paper.txt/bibliografía) enfocado en trazabilidad y verificación.
% Enlaza con la arquitectura (Sec. 3) donde aplicas verificación/hashes.

% =====================================================================
% 3. ARQUITECTURA DEL SISTEMA
% Qué incluir y cómo evaluarlo:
% - Diagrama de alto nivel: módulos (FIS, PSO, HC, UI, almacenamiento) y flujos de datos.
% - Contratos de I/O: qué datos produce/consume cada módulo y formatos.
% - Puntos de verificación: dónde se generan hashes/artefactos verificables.
% - UI: vistas clave y cómo disparan el pipeline.
% Evidencia: referencia a figuras `arquitectura.pdf` y `frontend_interface.png`.
% =====================================================================
% 3-arquitectura-sistema.tex
\section{Arquitectura del sistema}\label{sec:arquitectura}

\subsection{Visión de alto nivel}\label{subsec:arch-highlevel}
% Describe módulos y contratos. Inserta la figura `images/arquitectura.pdf`.

\subsection{Módulo de inferencia difusa (FIS)}\label{subsec:arch-fis}
% Entradas (p.ej. densidad/flujo/velocidad simulada), MFs, reglas, salida crisp de congestión.

\subsection{Módulo de optimización (PSO)}\label{subsec:arch-pso}
% Partícula = tiempos semafóricos; evaluación contra FIS; realimentación de impacto.

\subsection{Módulo de análisis (HC)}\label{subsec:arch-hc}
% Clustering 1D sobre salida crisp; stats de clusters para reporting.

\subsection{UI y visual analytics}\label{subsec:arch-ui}
% Interfaz: selección de gráfico y ejecución con datos sintéticos (Chart.js).
% Inserta la figura `images/frontend_interface.png`.

\subsection{Almacenamiento verificable y registro de ejecuciones}\label{subsec:arch-storage}
% Esquema de almacenamiento: IoI, metadatos de corrida, hashes/verificación.
% Indica en qué etapas se calculan/guardan y cómo se validan.

% =====================================================================
% 4. METODOLOGÍA Y FORMULACIÓN
% Qué incluir y cómo evaluarlo:
% - Datos/escenarios: define dataset sintético, supuestos y número de sensores.
% - FIS: universos, funciones de pertenencia (MFs), reglas, defuzzificación.
% - PSO: codificación de partículas, límites, función objetivo con penalizaciones.
% - HC: distancia, linkage (p. ej., Ward), umbral de corte, extracción de stats.
% - Métricas: define fórmulas; ej. mejora % = (orig - opt)/orig.
% Evidencia: enlaza con figuras `fuzzy_membership.png`, `pso_curve.png`, `hierarchical_clustering.png`.
% =====================================================================
% 4-metodologia-formulacion.tex
\section{Metodología y formulación}\label{sec:metodologia}

\subsection{Modelo de datos y escenarios}\label{subsec:meto-datos}
% Datos sintéticos por sensor (índice de sensor como eje canónico); supuestos y generación.

\subsection{FIS de congestión: MFs, reglas y defuzzificación}\label{subsec:meto-fis}
% Define universos, MFs (4), operadores, inferencia y defuzzificación.
% Inserta `images/fuzzy_membership.png` y referencia cada MF.

\subsection{PSO para tiempos semafóricos}\label{subsec:meto-pso}
% Codificación de partículas, límites min/max de verde/rojo, restricciones de ciclo.
% Función objetivo: minimizar congestión agregada con penalizaciones por restricciones.
% Inserta `images/pso_curve.png`; explica interpretación del paisaje de fitness.

\subsection{Clustering jerárquico (HC) sobre salida crisp}\label{subsec:meto-hc}
% Distancia, linkage (Ward), umbral de disimilitud; cómo se obtienen stats por cluster.
% Inserta `images/hierarchical_clustering.png`.

\subsection{Métricas de evaluación}\label{subsec:meto-metricas}
% Define explícitamente: índice de congestión, % de mejora ((orig - opt)/orig),
% consistencia (correlación orig vs opt), y viabilidad (respeto a tiempos).

% =====================================================================
% 5. IMPLEMENTACIÓN
% Qué incluir y cómo evaluarlo:
% - Estructura de carpetas/módulos: dónde viven FIS, PSO, HC y utilidades.
% - Orquestación: flujo de ejecución y dependencias entre módulos.
% - Parámetros clave: defaults y cómo se exponen/cambian.
% - UI y charts: eventos, actualización de gráficos y fuentes de datos.
% Evidencia: enlaza con archivos/código y capturas `frontend_interface.png`.
% =====================================================================
% 5-implementacion.tex
\section{Implementación}\label{sec:implementacion}

\subsection{Estructura de código y orquestación}\label{subsec:impl-codigo}
% Resume carpetas/módulos (según codigo.txt) y cómo se integran en el pipeline.

\subsection{Detalles clave por módulo}\label{subsec:impl-modulos}
% FIS: MFs y reglas; PSO: inicialización, límites y evaluación; HC: parámetros.
% Concreta constantes/funciones públicas relevantes (según codigo.txt).

\subsection{Interfaz gráfica y flujo de ejecución}\label{subsec:impl-ui}
% Carga de datos sintéticos; radios para elegir gráfico; updateChart(type) con Chart.js.
% Inserta captura `images/frontend_interface.png`.

\subsection{Gráficos generados y origen de datos}\label{subsec:impl-graficos}
% Lista cada gráfico (Bar, Line, Pie, Stacked, Scatter y=x, Grouped Bar, Improvement%).
% Define ejes/series y la fuente de datos (impact.*, optimization.*) y tooltips/leyendas.

\subsection{Reproducibilidad y configuración}\label{subsec:impl-reprod}
% Seeds, parámetros por defecto, archivos de configuración, logging y persistencia.

% =====================================================================
% 6. EXPERIMENTOS Y RESULTADOS
% Propósito: diseño, parámetros, resultados visuales, ablaciones y discusión.
% Nota: relacionar métricas con objetivos y limitaciones.
% =====================================================================
% 6-experimentos-resultados.tex
\section{Experimentos y resultados}\label{sec:resultados}

\subsection{Diseño experimental}\label{subsec:exp-diseno}
% Escenarios con datos simulados (normal, moderado, severo), número de sensores, iteraciones de PSO.

\subsection{Parámetros y setup}\label{subsec:exp-setup}
% Hiperparámetros PSO, límites semafóricos, configuración FIS y HC.

\subsection{Resultados visuales (gráficos 1–7)}\label{subsec:exp-graficos}
% Presentar y describir cada gráfico según su función (comparación, tendencia, distribución de impacto, etc.).
% Destacar patrones: puntos bajo y=x (mejora), asignaciones de tiempos viables, sensores con baja mejora.

\subsection{Ablaciones y sensibilidad}\label{subsec:exp-ablaciones}
% Sin PSO (solo FIS), sensibilidad a MFs o umbral HC, impacto del tamaño de enjambre.

\subsection{Discusión de limitaciones}\label{subsec:exp-limitaciones}
% Datos sintéticos; asumir movilidad y demanda; cómo afectaría calibración con datos reales.

% =====================================================================
% 7. INTEGRACIÓN Y ANÁLISIS
% Propósito: conectar FIS→PSO→HC y leer resultados a nivel operativo.
% Enlace con teoría (Kerner) y despliegue escalable/descentralizado.
% =====================================================================
% 7-integracion-analisis.tex
\section{Integración y análisis}\label{sec:integracion}
% Sección puente para “conectar” formalmente todo: cómo FIS→PSO→HC cierran el loop.
% Relación entre teoría (Kerner) y práctica (FIS+PSO); lectura por clusters para decisiones operativas a nivel tramo.
% Implicancias para escalabilidad y despliegue descentralizado con almacenamiento verificable.

% =====================================================================
% 8. CONCLUSIONES Y TRABAJO FUTURO
% Propósito: sintetizar hallazgos y proponer líneas de continuidad.
% Nota: conectar con objetivos planteados y métricas alcanzadas.
% =====================================================================
% 8-conclusiones-futuro.tex
\section{Conclusiones y trabajo futuro}\label{sec:conclusiones}
% Resumen de hallazgos, valor del pipeline y de la analítica (gráficos/HC) para explicar y priorizar.
% Futuro: calibración con datos reales, integración con LoRaWAN/edge, validación A/B, y robustez multi-intersección.


% =====================================================================
%                             AGRADECIMIENTOS
%  Reconoce financiación (CONACYT/FEEI), direcciones/mentores y apoyos técnicos.
%  Usa \section* para no numerar y agrégalo al índice si lo deseas.
% =====================================================================
% =====================================================================
% 9. AGRADECIMIENTOS
% Propósito: reconocer contribuciones de personas e instituciones.
% Sugerencia: mencionar financiamiento (CONACYT/FEEI) y apoyos académicos.
% =====================================================================
\section*{Agradecimientos}
\addcontentsline{toc}{section}{Agradecimientos}


% =====================================================================
%                               BIBLIOGRAFÍA
%  Estilo IEEEtran y base references.bib.
%  Recuerda: compila LaTeX → BibTeX → LaTeX → LaTeX para resolver citas.
% =====================================================================
\nocite{*}
\bibliographystyle{IEEEtran}
\bibliography{references}

\end{document}
