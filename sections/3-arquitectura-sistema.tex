% =====================================================================
% 3. ARQUITECTURA DEL SISTEMA
% Qué incluir y cómo evaluarlo:
% - Diagrama de alto nivel: módulos (FIS, PSO, HC, UI, almacenamiento) y flujos de datos.
% - Contratos de I/O: qué datos produce/consume cada módulo y formatos.
% - Puntos de verificación: dónde se generan hashes/artefactos verificables.
% - UI: vistas clave y cómo disparan el pipeline.
% Evidencia: referencia a figuras `arquitectura.pdf` y `frontend_interface.png`.
% =====================================================================
% 3-arquitectura-sistema.tex
\section{Arquitectura del sistema}\label{sec:arquitectura}

\subsection{Visión de alto nivel}\label{subsec:arch-highlevel}
% Describe módulos y contratos. Inserta la figura `images/arquitectura.pdf`.

\subsection{Módulo de inferencia difusa (FIS)}\label{subsec:arch-fis}
% Entradas (p.ej. densidad/flujo/velocidad simulada), MFs, reglas, salida crisp de congestión.

\subsection{Módulo de optimización (PSO)}\label{subsec:arch-pso}
% Partícula = tiempos semafóricos; evaluación contra FIS; realimentación de impacto.

\subsection{Módulo de análisis (HC)}\label{subsec:arch-hc}
% Clustering 1D sobre salida crisp; stats de clusters para reporting.

\subsection{UI y visual analytics}\label{subsec:arch-ui}
% Interfaz: selección de gráfico y ejecución con datos sintéticos (Chart.js).
% Inserta la figura `images/frontend_interface.png`.

\subsection{Almacenamiento verificable y registro de ejecuciones}\label{subsec:arch-storage}
% Esquema de almacenamiento: IoI, metadatos de corrida, hashes/verificación.
% Indica en qué etapas se calculan/guardan y cómo se validan.
