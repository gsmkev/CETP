% =====================================================================
% 1. INTRODUCCIÓN
% Propósito: contexto, problema, objetivos, contribuciones y estructura.
% Guía rápida: mantén los labels estables para referencias cruzadas.
% =====================================================================
% 1-introduccion.tex
\section{Introducción}\label{sec:intro}

\subsection{Motivación y contexto}\label{subsec:intro-motivacion}
% Congestión urbana, necesidad de control adaptativo y analítica explicable.
% Aterrizar en el caso de estudio y entorno de despliegue/validación.

\subsection{Planteamiento del problema}\label{subsec:intro-problema}
% Detección y cuantificación de congestión con salidas crisp; optimización de tiempos semafóricos; agrupamiento
% para análisis a nivel cluster de puntos de medición.

\subsection{Objetivos y preguntas de investigación}\label{subsec:intro-objetivos}
% Objetivo general + objetivos específicos (FIS para congestión, PSO para señalización, HC para análisis macro).
% Preguntas de investigación alineadas a métricas y escenarios de evaluación.

\subsection{Contribuciones}\label{subsec:intro-contribuciones}
% (1) Pipeline PSO+FIS+HC reproducible; (2) MFs y reglas fuzzy trazables; (3) Función objetivo y restricciones operativas;
% (4) Visual analytics con 7 gráficos estandarizados; (5) Arquitectura con almacenamiento verificable.

\subsection{Alcance y limitaciones}\label{subsec:intro-alcance}
% Simulaciones con datos sintéticos y pipeline orquestado; no se incluye calibración con datos reales de campo aún.

\subsection{Estructura del documento}\label{subsec:intro-estructura}
% Mapa de secciones 2–8.
