% =====================================================================
% 2. ANTECEDENTES Y TRABAJOS RELACIONADOS
% Qué incluir y cómo evaluarlo:
% - Fenomenología del tráfico: resume “tres fases” y cómo motivan métricas/objetivos.
% - FIS: ventajas de interpretabilidad; ejemplos en control semafórico.
% - PSO: aplicaciones a señalización; variables codificadas y funciones objetivo típicas.
% - HC: usos para segmentar congestión; criterios de corte y lectura operativa.
% - Almacenamiento verificable: trabajos sobre trazabilidad/reproducibilidad.
% Evidencia: cita 1–2 trabajos clave por subtema y enlaza con tu enfoque.
% =====================================================================
% 2-antecedentes-relacionados.tex
\section{Antecedentes y trabajos relacionados}\label{sec:related}

\subsection{Fenomenología del tráfico y teoría de tres fases}\label{subsec:related-kerner}
% Resumen de patrones, breakdown y “tres fases” (free, synchronized, wide moving jam) que motivan un índice de congestión
% y la necesidad de control adaptativo. Cita Kerner y enlaza con Sec. 4 (métricas).
% Ej.: La transición F→S y la propagación de atascos fundamentan FIS y HC.

\subsection{Lógica difusa en control de tráfico}\label{subsec:related-fis}
% MFs, reglas y defuzzificación en aplicaciones semafóricas; ventajas de interpretabilidad.

\subsection{Optimización por enjambre de partículas (PSO) aplicada a semáforos}\label{subsec:related-pso}
% Codificaciones típicas (tiempos de verde/rojo), función objetivo (congestión/cola/demoras), restricciones operativas.
% Contrasta con tu función objetivo en Sec. 4 (penalizaciones y límites).

\subsection{Clustering jerárquico para análisis de congestión}\label{subsec:related-hc}
% Uso de linkage/umbral para segmentar salidas crisp y extraer estadísticos por cluster.
% Justifica elección de linkage/umbral que utilizarás.

\subsection{Almacenamiento verificable y reproducibilidad}\label{subsec:related-storage}
% Breve estado del arte (según referencias de paper.txt/bibliografía) enfocado en trazabilidad y verificación.
% Enlaza con la arquitectura (Sec. 3) donde aplicas verificación/hashes.
