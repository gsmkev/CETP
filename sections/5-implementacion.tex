% =====================================================================
% 5. IMPLEMENTACIÓN
% Qué incluir y cómo evaluarlo:
% - Estructura de carpetas/módulos: dónde viven FIS, PSO, HC y utilidades.
% - Orquestación: flujo de ejecución y dependencias entre módulos.
% - Parámetros clave: defaults y cómo se exponen/cambian.
% - UI y charts: eventos, actualización de gráficos y fuentes de datos.
% Evidencia: enlaza con archivos/código y capturas `frontend_interface.png`.
% =====================================================================
% 5-implementacion.tex
\section{Implementación}\label{sec:implementacion}

\subsection{Estructura de código y orquestación}\label{subsec:impl-codigo}
% Resume carpetas/módulos (según codigo.txt) y cómo se integran en el pipeline.

\subsection{Detalles clave por módulo}\label{subsec:impl-modulos}
% FIS: MFs y reglas; PSO: inicialización, límites y evaluación; HC: parámetros.
% Concreta constantes/funciones públicas relevantes (según codigo.txt).

\subsection{Interfaz gráfica y flujo de ejecución}\label{subsec:impl-ui}
% Carga de datos sintéticos; radios para elegir gráfico; updateChart(type) con Chart.js.
% Inserta captura `images/frontend_interface.png`.

\subsection{Gráficos generados y origen de datos}\label{subsec:impl-graficos}
% Lista cada gráfico (Bar, Line, Pie, Stacked, Scatter y=x, Grouped Bar, Improvement%).
% Define ejes/series y la fuente de datos (impact.*, optimization.*) y tooltips/leyendas.

\subsection{Reproducibilidad y configuración}\label{subsec:impl-reprod}
% Seeds, parámetros por defecto, archivos de configuración, logging y persistencia.
