% =====================================================================
% 6. EXPERIMENTOS Y RESULTADOS
% Propósito: diseño, parámetros, resultados visuales, ablaciones y discusión.
% Nota: relacionar métricas con objetivos y limitaciones.
% =====================================================================
% 6-experimentos-resultados.tex
\section{Experimentos y resultados}\label{sec:resultados}

\subsection{Diseño experimental}\label{subsec:exp-diseno}
% Escenarios con datos simulados (normal, moderado, severo), número de sensores, iteraciones de PSO.

\subsection{Parámetros y setup}\label{subsec:exp-setup}
% Hiperparámetros PSO, límites semafóricos, configuración FIS y HC.

\subsection{Resultados visuales (gráficos 1–7)}\label{subsec:exp-graficos}
% Presentar y describir cada gráfico según su función (comparación, tendencia, distribución de impacto, etc.).
% Destacar patrones: puntos bajo y=x (mejora), asignaciones de tiempos viables, sensores con baja mejora.

\subsection{Ablaciones y sensibilidad}\label{subsec:exp-ablaciones}
% Sin PSO (solo FIS), sensibilidad a MFs o umbral HC, impacto del tamaño de enjambre.

\subsection{Discusión de limitaciones}\label{subsec:exp-limitaciones}
% Datos sintéticos; asumir movilidad y demanda; cómo afectaría calibración con datos reales.
