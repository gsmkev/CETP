% =====================================================================
% 4. METODOLOGÍA Y FORMULACIÓN
% Qué incluir y cómo evaluarlo:
% - Datos/escenarios: define dataset sintético, supuestos y número de sensores.
% - FIS: universos, funciones de pertenencia (MFs), reglas, defuzzificación.
% - PSO: codificación de partículas, límites, función objetivo con penalizaciones.
% - HC: distancia, linkage (p. ej., Ward), umbral de corte, extracción de stats.
% - Métricas: define fórmulas; ej. mejora % = (orig - opt)/orig.
% Evidencia: enlaza con figuras `fuzzy_membership.png`, `pso_curve.png`, `hierarchical_clustering.png`.
% =====================================================================
% 4-metodologia-formulacion.tex
\section{Metodología y formulación}\label{sec:metodologia}

\subsection{Modelo de datos y escenarios}\label{subsec:meto-datos}
% Datos sintéticos por sensor (índice de sensor como eje canónico); supuestos y generación.

\subsection{FIS de congestión: MFs, reglas y defuzzificación}\label{subsec:meto-fis}
% Define universos, MFs (4), operadores, inferencia y defuzzificación.
% Inserta `images/fuzzy_membership.png` y referencia cada MF.

\subsection{PSO para tiempos semafóricos}\label{subsec:meto-pso}
% Codificación de partículas, límites min/max de verde/rojo, restricciones de ciclo.
% Función objetivo: minimizar congestión agregada con penalizaciones por restricciones.
% Inserta `images/pso_curve.png`; explica interpretación del paisaje de fitness.

\subsection{Clustering jerárquico (HC) sobre salida crisp}\label{subsec:meto-hc}
% Distancia, linkage (Ward), umbral de disimilitud; cómo se obtienen stats por cluster.
% Inserta `images/hierarchical_clustering.png`.

\subsection{Métricas de evaluación}\label{subsec:meto-metricas}
% Define explícitamente: índice de congestión, % de mejora ((orig - opt)/orig),
% consistencia (correlación orig vs opt), y viabilidad (respeto a tiempos).
